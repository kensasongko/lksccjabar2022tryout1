\documentclass{article}

\usepackage[english]{babel}

\usepackage[letterpaper,top=2cm,bottom=2cm,left=3cm,right=3cm,marginparwidth=1.75cm]{geometry}

\usepackage[colorlinks=true, allcolors=blue]{hyperref}

\title{LKS Cloud Computing Provinsi Jawa Barat 2022 - Tryout 1}
\author{}

\begin{document}
\maketitle

\section*{Task 1}
\begin{enumerate}
\item Create a Hosted Zone in Route53 and update your domain's NS records to use Route 53 name servers.\\
Reference: \href{https://docs.aws.amazon.com/Route53/latest/DeveloperGuide/Welcome.html}{Route 53 Documentation}
\item Ensure your domain has been configured correctly in Route 54 (e.g., Create an A record and check the result with ping/dig/nslookup).
\end{enumerate}

\section*{Task 2}
Follow the following tutorial to create CRUD API with Lambda and DynamoDB:
\href{https://docs.aws.amazon.com/apigateway/latest/developerguide/http-api-dynamo-db.html}{CRUD Tutorial}

\section*{Task 3}
\begin{enumerate}
\item Create a bucket and enable static website hosting. The bucket name must contain: "LKSCC2022-[Your LKS ID]"\\
Reference: \href{https://docs.aws.amazon.com/AmazonS3/latest/userguide/Welcome.html}{S3 Documentation}
\item Create a file named index.html, the body must contain your LKS ID.\\
Reference: \href{https://www.w3schools.com/html/html_basic.asp}{HTML Basic Example} 
\item Upload index.html to the bucket.
\item Ensure that you can open the file via the static website hosting URL.
\item Create CloudFront distribution with static website hosting URL as the origin.\\
Reference: \href{https://docs.aws.amazon.com/AmazonCloudFront/latest/DeveloperGuide/Introduction.html}{CloudFront documentation}
\item Ensure your CloudFront distribution content matches the S3 bucket content.
\end{enumerate}

\section*{Task 4}
\begin{enumerate}
\item Create a VPC with 2 availability zones (AZs), each AZ must have a public subnet and a private subnet (NAT).\\
Reference: \href{https://docs.aws.amazon.com/vpc/latest/userguide/what-is-amazon-vpc.html}{VPC documentation}
\item Create Elastic Load Balancer with the following specification:
\begin{itemize}
\item Type: Application Load Balancer
\item Security Group: Allow HTTP and HTTPS from anywhere
\end{itemize}
Reference: \href{https://docs.aws.amazon.com/elasticloadbalancing/latest/userguide/what-is-load-balancing.html}{Elastic Load Balancer documentation}
\item Create an EC2 instance,install nginx.\\
Reference: \href{https://docs.aws.amazon.com/AWSEC2/latest/UserGuide/concepts.html}{EC2 Documentation}
\item Create an AMI for Autoscaling Group using the EC2 instance you have created previously.\\
Reference: \href{https://docs.aws.amazon.com/toolkit-for-visual-studio/latest/user-guide/tkv-create-ami-from-instance.html}{Create an AMI from EC2 Instance}
\item Create an Austoscaling Group with the following specifications:
\begin{itemize}
\item Subnet: Private
\item Instance Type: t2.micro
\item Desired Capacity: 2
\item Minimum Capacity: 2
\item Maximum Capacity: 2
\end{itemize}
\item Forward Application Load Balancer's incoming traffic to Autoscaling Group
\item Ensure you can see the default NGINX's page from the internet by opening Application Load Balancer's URL.
\item (Optional) Ensure you can create RDS MySQL instance.\\
Reference: \href{https://docs.aws.amazon.com/AmazonRDS/latest/UserGuide/CHAP_GettingStarted.html}{RDS Documentation}
\item (Optional) Ensure you can create ElastiCache Redis cluster.\\
Reference: \href{https://docs.aws.amazon.com/AmazonElastiCache/latest/red-ug/WhatIs.html}{ElastiCache Documentation}
\end{enumerate}

\end{document}
